\section{Diskusjon}

\subsection{Metode og måleforutsetninger}
I krets 1 benyttet vi både maskeanalyse og nodeanalyse for å beregne oss fram til ulike kjente verdier i et motstandsnettverk med uavhengige spenningskilder. Krets 2 er for det meste lik som krets 1, men her benyttet vi maskeanalyse som eneste metode for å sette opp maskelikningene. For å utføre dette i praksis benyttet vi oss av  instrumentene og komponentene i tabell~\ref{tab:instrumentliste} og~\ref{tab:komponentliste}. Nå vil de praktiske resultatene diskuteres i forhold til de teoretiske resultatene.

\subsection{Feilkilder}
Elektriske kretser er alltid forbundet med ulike feilkilder som kan påvirke måleresultatene. 
Disse feilkildene er hovedårsaken til at praktiske målinger avviker fra teoretiske verdier. Selv når resultatene viser god overensstemmelse, vil de mindre avvikene kunne forklares gjennom slike feilkilder. I denne rapporten vurderes de mest relevante feilkildene for krets~1, krets~2 og krets~3. De omfatter kontaktmotstand og dårlige koblinger, spenningskildens egenskaper, måleinstrumentets påvirkning, restspenninger i kretsen, samt generell måleusikkerhet og oppløsning. 

\paragraph{Kontaktmotstand og dårlige koblinger.}
I koblingsbrett, ledninger og komponentoverganger vil kontaktmotstand oppstå. Kontaktmotstanden er den elektriske motstanden som oppstår i grenseflaten mellom to ledere som er koblet sammen~\cite{kontaktmotstandSNL}. Denne vil være variabel i hvert kontaktpunkt og vil garantere ubalanser i målinger i hele kretsen. Dårlige kontaktpunkter vil gi økt motstand, og dette særlig i lavstrømsgrener. Ser vi resultater i krets 1 (tabell~\ref{tab:results1}) ser vi at de fleste målinger er innenfor $1\%$, men kontaktmotstand og dårlige koblinger kan forklare avviket på $1.25\%$ i strøm $I_{R2}$.


\paragraph{Spenningskilden.}
I teoretiske utregninger antar vi at den interne motstanden i spenningskilden er lik null for simplisiteten av kretsen. Dette lager en reell feilkilde der det vil være en liten, ukjent indre motstand i spenningskildekomponenten. I tillegg vil spenningskilden ha små variasjoner i utgangsspenning over tid. Dette påvirker hvordan strømmen over motstand $R_1$ og $R_2$ vil variere og ikke være perfekt i forhold til den teoretiske utregningen. I tillegg kan vi trekke dette under nullspenningen i node B. Når de to uavhengige spenningskildene produserer inkonsistente spenninger vil det påvirke spenningen i node B. I krets 3 kan vi se på formelen for $V_{out}$ (ligning~\ref{eq:spenning_i_wheatstone}):

\[
V_{out} = V_{in} \left( \frac{R_3}{R_1+R_3} - \frac{R_4}{R_2+R_4} \right)
\]
\noindent
Antar vi at resistansen i hver motstand er konstant vil dette gi oss et uttrykk der spenningen inn er proporsjonal med den målte spenningen.
\[
V_{out} = V_{in} \cdot x, \qquad x = \frac{R_3}{R_1+R_3} - \frac{R_4}{R_2+R_4}
\]

\noindent
Spenningskilden produserte spenning er proporsjonal med den målte spenningen og siden spenningskilden er en feilkilde i seg selv vil dette gi et grunnlag for å forklare avvik i målinger som omfavner $V_{12}$.


\paragraph{Måleinstrumentet.}
Multimeteret er ikke noe vi tar hensyn til i teoretiske utregninger siden det spiller ikke en rolle i en krets. Multimeterets eneste rolle i en krets er å parallellkobles i to punkter av interesse. Multimeteret har en høy indre motstand for å forsøke å ikke påvirke kretsen. Uansett, vil den indre motstanden aldri være uendelig og komponenten vil dra noe strøm. I tillegg vil oppløsningen på instrumentet forhindre de mest nøyaktige målingene. Dette kan ha innspill på målingene til en viss grad, og vi kan si at det er noe av grunnen til avvik i alle praktiske målinger.



\paragraph{Restspenninger.}
Restspenninger oppstår på grunn av kontaktmotstand og dårlige koblinger. Det vil praktisk si at kontaktmotstandene vil være variable og variere, og dermed lage små restspenningsforskjeller (jordpotensialforskjeller) overalt i kretsen. Dette vil påvirke nullspenningen i node B spesielt, men det vil også ha innvirkninger på alle tre kretser.


\paragraph{Måleusikkerhet og oppløsning.}
Multimeteret vil ha en måleusikkerhet på grunn av dens oppløsning i hva den kan måle. Dette bidrar til at små spenninger og strømmer har høy relativ feil. Et multimeter med nøyaktighet $\pm 0.5\%$ gir usikkerhet på rundt $\pm 0.06\text{V}$ ved $12\text{V}$ måling. I tillegg for millivolt-nivåer (som $V_B = 0.028\text{V}$), er usikkerheten svært stor. Dette vil påvirke resultatene våre og noe av avvikene kan forklares av denne feilkilden.

\vspace{1em}
\noindent
Samlet sett forklarer disse feilkildene de små forskjellene mellom teoretiske og eksperimentelle verdier i forsøket. De største bidragene antas å være kontaktmotstand og måleusikkerhet, mens variasjon i spenningskildene og restspenninger har mindre, men merkbar innvirkning på nøyaktigheten.







\subsection{Resultater knyttet til teori}

Dette delkapitlet sammenligner de målte verdiene fra laboratorieforsøket med de teoretiske beregnet utledet i kapittel~\ref{sec:toeri}. Hensikten er å vurdere om resultatene følger de forventede sammenhengene mellom Ohms lov, Kirchhoffs lover, og teorien for maske- og nodeanalyse samt Wheatstone-broen. Under vil det drøftes resultater opp mot teori for hver krets med feilkilder i fokus der avviket er større enn $\pm 1\%$.

\subsubsection{Krets 1}
Metoden for utregning brukt her er maske- og node-analyse. Begge metodene gir $I_{R1} = I_{R2} = 8.00\text{mA}$ og $I_{R3} = 0\text{A}$. De målte resultatene varierer litt og gir alle litt avvik fra den teoretiske verdien. Strøm $I_{R2}$ avviker mest i hele tabell~\ref{tab:results1} med over $1\%$, og resten under. Avviket i $I_{R2}$ kan forklares av flere feilkilder, som kontaktmotstand og dårlige koblinger. Andre feilkilder som spenningskilden vil også kunne spille inn her, da den kan produsere spenninger med feilmarginer. I tillegg kan vi se at i krets 1 er motstanden $R_2$ koblet i serie fra et nodepotensial $-12\text{V}$ i node C til et nodepotensial $0\text{V}$ i node B. Våre målinger for $V_C$ og $V_B$ samsvarer ikke med dette, og det kan være en grunn for at strømmen i $R_2$ har mer avvik. Grunnet feilkilder kan vi trygt si at $I_{R2}$ ligger innenfor en akseptabel avstand fra den teoretiske verdien. Tar vi for oss regnestykket for strømmen gjennom motstanden $R_2$ med de praktiske målingene for spenninger får vi:
\[
I_{R2} = - \frac{V_C - V_B}{R_2}
\]
\[
I_{R2} = - \frac{-12.10\text{V} - 0.028\text{V}}{1500\Omega}
\]
\[
I_{R2} = 8.05\text{mA}
\]
\noindent
Dette viser at spenningen i node C bidrar til at $I_{R2} > 8.00\text{mA}$. $V_C$ har allerede et avvik på $0.83\%$ som er ganske stort, men akseptabelt siden det ligger innenfor $1\%$. Målingen for $V_C$ er tatt i parallell mellom selve spenningskilden ($-12\text{V}$) og jording ($0\text{V}$) og vil gjenspeile hva spenningskilden er stilt inn til. Derfor vil disse ujevne målingene direkte stamme fra spenningskilden og restspenninger som mulige feilkilder. Samtidig kan vi trekke $V_B$ under dette fordi den bygger på relasjonen mellom spenningen i node A og C og motstandene mellom, som er utledet og presentert i ligning~\ref{eq:KCLiB} som vist under. 
\[
\frac{V_A-V_B}{R_1} = \frac{V_B}{R_3} + \frac{V_B - V_C}{R_2}
\]
\noindent
Her ser vi tydelig at spenningen i node B vil avhenger av mange feilkilder som spiller inn på hvert enkeltelement i ligningen. I tillegg har hver motstand et feilmarginområde i sin resistanse på $\pm 1\%$. Dette fører til et teoretisk usikkerhetsområde på $\pm 80\text{mV}$. Den praktiske målingen er $+ 28\text{mV}$. Målingen er innenfor godkjente rammer.
\\[1em]
Vi kan trekke en konklusjon for målingene i krets 1 om at de ligger innenfor forsøkets satte godkjente grense på $1\%$ med noen unntak som kan forklares i grunnleggende kretsfeilkilder.

\subsubsection{Krets 2}
Krets 2 bygger videre på krets 1, men med en ekstra gren via motstanden $R_4$ mellom node A og C. 
Teoretiske beregninger ved maskeanalyse ga $I_{R4}=6.15\text{mA}$, mens målingen viste $6.21\text{mA}$, et avvik på $0.98\%$. 
Dette er innenfor komponentenes toleranse på $\pm1\%$, og samsvarer derfor med teorien.
\\[1em]
Vi kan sammenligne strømverdiene gjennom $R_1$ og $R_2$ mellom krets 1 og 2 (tabell~\ref{tab:results1} og~\ref{tab:results2}). Her ser vi at de holder seg nær identiske, begge rundt $8\text{mA}$. 
Dette viser at tilkoblingen av $R_4$ primært påvirker strømfordelingen mellom grenene, uten å endre nodepotensialene $V_A$, $V_B$ og $V_C$ merkbart. Dette ser vi også i nodeanalyse for node B i teorien (ligning~\ref{eq:KCLiB}):
\[
\frac{V_A-V_B}{R_1} = \frac{V_B}{R_3} + \frac{V_B - V_C}{R_2}
\]
\noindent
For å forklare avviket i målingen for $I_{R4}$ kan vi isolere oss til selve grenen mellom node A og C og se på strømmen. Avviket kan forklares av kontaktmotstand og interne spenningskildevariasjoner. Spenningene i nodene viser også god overensstemmelse med teori. Node B har en målt spenning på $0.028\text{V}$, som er godt innenfor beregnet område $\pm 80\text{mV}$ fra ligning~\ref{eq:VBminmax}. 
Avvikene mellom målt og teoretisk verdi er dermed konsistente med forventede feilkilder, og bekrefter at Kirchhoffs lover holder innenfor eksperimentell nøyaktighet.
\\[1em]
Samlet sett for krets 2 er målingene innenfor godkjente rammer. Kretsens resultater holder seg konsistent med resultatene fra krets 1. Et nært $1\%$ avvik i målingen $I_{R4}$ kan forklares av feilkilder.

\subsubsection{Krets 3}
I teorien utledet vi et uttrykk for spenningen i Wheatstone-broen (se ligning~\ref{eq:spenning_i_wheatstone}). Med potensiometre modellert diagonalt i broen utledet vi uttrykket (se ligning~\ref{eq:pot_i_wheatstone}):
\[
V_{out}(x) = V_{in}\frac{x - R}{x + R}
\]
\noindent
Denne gir maksimal følsomhet nær balansepunktet $\left( x \approx R \right)$. Ser vi på resultatene i tabell~\ref{tab:results3} målte vi spenningen i broen til å være $6.5\text{mV}$, da den teoretiske verdien er $0\text{V}$ så er dette godt innafor. Komponentene som påvirker $V_{out}$ kan vi lese av ligning~\ref{eq:spenning_i_wheatstone} til å være motstandene og spenningskilden i kretsen. Aktuelle feilkilder vil da være kontaktmotstand og dårlige koblinger i motstandene. Siden kretsen har fire slike motstander vil ubalansen påvirket av feilkildene bli større. I tillegg til at selve kildespenningen vil ha egne mulige feilkilder kan vi ikke lene oss på at $V_{in}$ i formelen er pålitelig heller. Restspenninger vil dannes i nodene i Wheatstone-broen og vil også påvirke målingen $V_{12}$. 
\\[1em]
Videre kan vi se på målingen $V_{12}|_{min}$ i tabell~\ref{tab:results3} og den har et avvik på $0.25\%$ som er godt innenfor. Siden dette er en spenningsmåling ser vi på de samme effektive feilkildene som i måling $V_{12}$. Videre mulige feilkilder vil tilknyttes potensiometerne i kretsen og vi må se på uttrykket for $V_{out}(x)$, der $x$ er motstanden i begge potensiometerne for å finne aktuelle feilkilder for potensiometerbroen. $V_{12}|_{min}$ innebærte å stille potensiometerne til $0$, potensiometerne er derfor ikke en veldig stor feilkilde utenom dens indre motstand. Målingen for $V_{12}|_{min}$ blir derfor nøyaktig. 
\\[1em] 
$V_{12}|_{maks}$ har et mye større og uakseptabelt avvik på $3.64\%$. Her må vi trekke fram de relevante feilkildene og teoretiske formlene for å besvare dette. Den teoretiske verdien baserer seg på at begge potensiometerne har maksimummotstand på nøyaktig $100\text{k}\Omega$, men det er ikke sikkert og derfor vil det bli en aktuell feilkilde i forsøket vårt. Ligning~\ref{eq:pot_i_wheatstone} kan gi grunnlag for å forklare den mulige motstandsfeilkilden:
\[
V_{out}(x) = V_{in} \cdot  \left(\frac{x - R}{x + R} \right)
\]
\noindent
Hvis vi skal få $V_{out}$ til å nærme seg den praktiske målingen $6.27\text{V}$ må vi anta at potensiometerne har lik, men avvikende verdi. For å finne $x$ tar vi for oss regnestykket:
\[
\begin{aligned}
  V_{out}(x) &= 6.27\text{V} \\
  x &\approx 105.22\text{k}\Omega
\end{aligned}
\]
Dette presenterer at potensiometerne må begge ha en egentlig resistansefeilmargin på $+5.22\%$. Som er realistisk, men dette forutser at hele feilkilden ligger i akkurat dette momentet. Det er i tillegg til potensiometerresistansen de samme aktuelle feilkildene her som i måling av $V_{12}|_{min}$. Derfor kan vi påstå at målingen er realistisk, men med en uakseptabel avviksprosent grunnet ulike mulige feilkilder.


\subsection{Hovedfunn opp mot hensikt}
Hensikten med forsøket var å belyse prinsippene bak maske- og nodeanalyse, samt studere hvordan en Wheatstone-målebru kan brukes til å måle små motstandsvariasjoner. 
Resultatene fra krets 1 og krets 2 viser at både maskeanalyse og nodeanalyse gir korrekte og konsistente teoretiske verdier for strømmer og nodepotensialer, der avvik kan forklares av mulige feilkilder. De fleste praktiske målingene ligger innenfor forventede avvik basert på komponenttoleranser og måleusikkerhet, og noen ligger utenfor. Målinger som ligger utenfor har vært argumentert for med feilkilder. Uansett, så bekrefter de praktiske måledataene at metodene beskriver kretsene på en presis måte.
\\[1em]
For Wheatstone-broen i krets 3 ble det observert et restsignal nær balansepunktet, samt en tydelig endring i $V_{out}$ når forholdet mellom motstandene ble justert. Dette stemmer godt med den teoretiske modellen, og demonstrerer hvordan broen er spesielt følsom for små endringer i motstandsverdi rundt balansepunktet.
\\[1em]
Samlet sett viser forsøket at både maske- og nodeanalyse fungerer som pålitelige verktøy for kretsanalyse, og at Wheatstone-broen er velegnet for presis måling av små motstandsvariasjoner, slik forsøkshensikten beskriver.



\subsection{Mulige bruksområder}
Maske- og nodeanalyse brukes bredt innen analyse, design og feilsøking av elektriske kretser. 
Metodene danner grunnlaget for å bestemme strøm- og spenningsfordeling i alt fra enkle motstandsnettverk til mer komplekse signalbehandlings- og sensorkretser, og inngår som standard verktøy i både undervisning og industrirettet kretskonstruksjon~\cite{aac_mesh, aac_node}.
\\[1em]
Wheatstone-broen er spesielt nyttig når små endringer i motstand skal måles presist. 
Broen gjør det mulig å omsette svært små variasjoner i resistans til målbare spenningsendringer, og brukes derfor i et stort spekter av sensorer og målesystemer~\cite{omega_wheatstone}.
\\[1em]
Typiske anvendelser inkluderer strekkmålere i vektceller, hvor mekanisk deformasjon påvirker motstanden, samt temperaturavhengige motstander (NTC/PTC) og trykk- eller posisjonssensorer. 
I slike systemer gir Wheatstone-broen høy følsomhet og god stabilitet, noe som gjør den egnet i presisjonsmåleinstrumenter og industrielle overvåkingsapplikasjoner.



\subsection{Forbedringer og videre arbeid}
En naturlig forbedring av forsøket ville være å måle de faktiske resistansverdiene til alle motstandene som inngår i kretsene. Dette ville gjort det mulig å beregne nye teoretiske strøm- og spenningsverdier med reelle komponentdata, og dermed undersøke om avvikene hovedsakelig skyldes toleranser eller andre feilkilder. Dette kunne også gitt et bedre grunnlag for å forklare avviket i $V_{12}|_{maks}$ i krets 3, ved å analysere uttrykket for $V_{out}(x)$ med potensiometerets faktiske motstandsverdi.
\\[1em]
En annen forbedring ville være å karakterisere spenningskilden. I dette forsøket ble den interne motstanden og nøyaktigheten til kilden antatt, men ikke målt. Ved å måle kildeimpedans og faktisk utgangsspenning under last kunne mer presise simuleringer utføres, og avvikene i både krets 1 og 2 kunne vurderes mer nøyaktig.
\\[1em]
For Wheatstone-broen i krets 3 kunne forsøket vært utvidet ved å måle $V_{out}$ for flere posisjoner på potensiometeret og deretter plotte den praktiske kurven. En slik kurve kunne sammenlignes direkte med den teoretiske modellen vist i figur~\ref{fig:wheatstone-graph}, og dette ville gitt innsikt i hvor følsom broen er nær balansepunktet og hvor avviket øker mot endeposisjonene.

\paragraph{Videre arbeid.}  
Dette forsøket har introdusert og anvendt prinsippene for både node- og maskeanalyse, samt demonstrert funksjonen til en Wheatstone-bro. Et naturlig videre arbeid vil være å anvende disse analysemetodene på mer komplekse nettverk, for eksempel kretser som inneholder forsterkere eller flere koblingsnoder, der systematisk kretsanalyse blir enda viktigere.
\\[1em]
Videre kunne Wheatstone-broen i krets 3 vært brukt som utgangspunkt for et mer praktisk målesystem. Wheatstone-broen brukes ofte i sensorer der små endringer i motstand må omsettes til målbare elektriske signaler, slik som i \textit{strain gauges} (strekkmålere) og vektceller \cite{omega_wheatstone}. Et aktuelt videre eksperiment kunne derfor vært å erstatte potensiometeret med en strekkmåler, og undersøke hvordan mekanisk deformasjon gir utslag i brospenningen. Dette ville gi en direkte kobling mellom teoretisk kretsanalyse og reelle, fysiske måleapplikasjoner.