\section*{Sammendrag}

Denne laboratorieøvelsen hadde som formål å undersøke og anvende maske- og nodeanalyse for å beregne strømmer og spenninger i motstandsnettverk med uavhengige spenningskilder, samt å studere funksjonen til en Wheatstone-målebru. For krets 1 og 2 ble teoretiske verdier bestemt ved hjelp av Kirchhoffs lover, og disse ble sammenlignet med praktiske målinger gjennomført med digitalt multimeter. Resultatene viste gjennomgående små avvik, hovedsakelig under $1\%$, noe som er i tråd med motstandenes toleranser og forventet måleusikkerhet. Krets 3 demonstrerte hvordan Wheatstone-broen kan balanseres og hvordan små variasjoner i motstandsforhold gir utslag i utgangsspenningen. Her ble spesielt den høye følsomheten nær balansepunktet tydelig observert. Identifiserte avvik ble forklart gjennom feilkilder som kontaktmotstand, kildevariasjoner og instrumentoppløsning. Øvelsen ga dermed både teoretisk og praktisk forståelse av kretsanalysemetoder og viste hvordan Wheatstone-broen kan benyttes i presisjonsmåling og sensorsystemer.