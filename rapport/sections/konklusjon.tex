\section{Konklusjon}
Formålet med forsøket var å belyse prinsippene bak maske- og nodeanalyse, i tillegg til å studere funksjonen til en Wheatstone-målebru. De praktiske måleresultatene samsvarer med teorien og viser at både maske- og nodeanalyse er pålitelige metoder for å analysere elektriske kretser. Avvikene er akseptable, men kan også forklares av feilkilder.
\\[1em]
Wheatstone-broen viste en tydelig sammenheng mellom ubalanse i motstandsforholdene og utgangsspenningen. Målingene viser at broen er mest følsom nær balansepunktet, slik teorien tilsier. Avvikene i maksimal utgangsspenning kan knyttes til praktiske forhold som internmotstand i spenningskilden og feilmargin i potensiometermotstanden.

Samlet sett oppfylte forsøket hensikten. Arbeidet ga både teoretisk og praktisk innsikt i hvordan elektriske kretsanalyser utføres, og hvordan Wheatstone-broen kan benyttes til nøyaktige resistansmålinger. Videre arbeid kan omfatte måling av faktiske komponentverdier, karakterisering av spenningskilde og måleinstrument, samt bruk av broen sammen med temperatur- eller streinmålesensorer i et komplett målesystem.