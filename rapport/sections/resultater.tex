\section{Gjennomføring med måleresultater}

\subsection{Krets 1}
\vspace{-1em}
\begin{table}[H]
\centering
\caption{Sammenlikning av teoretiske og målte verdier av komponenter i krets 1}
\vspace{1em}
\begin{tabular}{lccc}
\toprule
\textbf{Komponent} & \textbf{Teoretisk verdi} & \textbf{Målt verdi} & \textbf{Avvik [\%]} \\
\midrule
Strøm $I_{R1}$ & 8.00mA & 7.95mA & 0.63\% \\
Strøm $I_{R2}$ & 8.00mA & 8.10mA & 1.25\% \\
Strøm $I_{R3}$ & 0A & 0.019mA & --- \\
Spenning $V_{A}$ & 12V & 12.01V & 0.08\% \\
Spenning $V_{B}$ & 0V & 0.028V & --- \\
Spenning $V_{C}$ & -12V & -12.10V & 0.83\% \\

\bottomrule
\end{tabular}
\label{tab:results1}
\end{table}


\subsubsection{Fremgangsmåte}
\label{subsec:krets1_fremgangsmåte}
For å koble opp krets 1 som vist i figur \ref{fig:krets1} må vi kunne koble positiv og negativ spenningskilder. $V_1$ kobles med negativ terminal til jording, og den positive terminalen til motstand $R_1$. $V_2$ derimot må kobles motsatt, der den positive terminalen kobles i jording, og negativ terminal til motstand $R_2$. $R_1$ og $R_2$ kobles sammen til $R_3$ som kobles til jording. Et viktig moment her er at spenningskilden er allerede koblet til jording og trenger ikke en slik kobling på koblingsbrettet, ettersom det vil kortslutte hele kretsen.

\subsubsection{Kommentar til resultater}
De målte verdiene viser svært små avvik fra de teoretiske beregningene. Alle målinger utenom én ligger godt innenfor komponentenes toleranse på 1\%. $I_{R2}$ skiller seg ut. Avviket i $V_B$ er på 28mV, som ligger innenfor det teoretisk beregnede intervallet $[-80\text{mV}, 80\text{mV}]$ (Ligning \ref{eq:VBminmax}).





\subsection{Krets 2}
\vspace{-1em}
\begin{table}[H]
\centering
\caption{Sammenlikning av teoretiske og målte verdier av komponenter i krets 2}
\vspace{1em}
\begin{tabular}{lccc}
\toprule
\textbf{Komponent} & \textbf{Teoretisk verdi} & \textbf{Målt verdi} & \textbf{Avvik [\%]} \\
\midrule
Strøm $I_{R1}$ & 8.00mA & 8.03mA & 0.38\% \\
Strøm $I_{R2}$ & 8.00mA & 8.07mA & 0.88\% \\
Strøm $I_{R3}$ & 0A & 0.021mA & --- \\
Strøm $I_{R4}$ & 6.15mA & 6.21mA & 0.98\% \\
Spenning $V_{A}$ & 12V & 12.01V & 0.08\% \\
Spenning $V_{B}$ & 0V & 0.028V & --- \\
Spenning $V_{C}$ & -12V & -12.09V & 0.75\% \\
\bottomrule
\end{tabular}
\label{tab:results2}
\end{table}

\subsubsection{Fremgangsmåte}
I denne kretsen følger samme fremgangsmåte som i krets 1 (avsnitt \ref{subsec:krets1_fremgangsmåte}). I tillegg må vi koble node A og node C sammen med en motstand $R_4$ mellom.

\subsubsection{Kommentar til resultater}
Resultatene viser god overensstemmelse mellom teoretiske og målte verdier. Alle strømmer og spenninger avviker mindre enn 1\%, noe som er innenfor forventet måleusikkerhet og motstandenes toleranse. 
Det svake avviket i $I_{R4}$ og $V_C$ kan skyldes små forskjeller i motstandsverdiene eller variasjoner i spenningskildene.



\subsection{Krets 3}
\vspace{-1em}
\begin{table}[H]
\centering
\caption{Sammenlikning av teoretiske og målte verdier av komponenter i krets 3}
\vspace{1em}
\begin{tabular}{lccc}
\toprule
\textbf{Komponent} & \textbf{Teoretisk verdi} & \textbf{Målt verdi} & \textbf{Avvik [\%]} \\
\midrule
Spenning $V_{12}$ & 0V & 6.5mV & --- \\
Spenning $V_{12} |_{min}$ & -12V & -11.97V & 0.25\% \\
Spenning $V_{12} |_{maks}$ & 6.05V & 6.27V & 3.64\% \\
\bottomrule
\end{tabular}
\label{tab:results3}
\end{table}


\subsubsection{Fremgangsmåte}
Denne kretsen kobles opp med en spenningsgenerator som har positiv terminal til koblinsbrettet og negativ terminal til jording. Fra spenningskildekoblingen på brettet kobler vi til et nodepunkt hvor det skal etableres en parallellkobling av to grener. Den ene grenen har motstandene $R_5$ og $R_6$ i seriekobling, og den andre grenen har motstandene $R_7$ og $R_8$ i seriekobling. Noden der parallellkoblingen samles går til jording på brettet.

\subsubsection{Kommentar til resultater}
Balansespenningen $V_{12}$ viser et akseptabelt avvik på $6.5\text{mV}$. I tillegg har minimumsverdien en god måling og samsvarer med teoretiske kalkulasjoner. Derimot har målingen $V_{12}|_{\text{maks}}$ et uakseptabelt avvik på $3.64\%$, dette overskrider minimumstoleransemålingen på $\pm 1\%$.
